\documentclass[titlepage]{article}

\begin{document}
	\title{Linux Learning}
	\maketitle
	
	\section{Commands}
	  \subsection{information}
	    \begin{itemize}
	    	\item id -- account info
	    	\item uname -- system info
	    	\item who -- login session info
	    \end{itemize}
    
      \subsection{program location}
        \begin{itemize}
        	\item type -- bash built in, returns with program type
        	\item which -- search from PATH
        	\item locate -- search from the internal db
        \end{itemize}
    
    \section{Environment Variables}
      \subsection{h}
        \begin{itemize}
        	\item HISTFILE -- file path of the command history
        	\item HISTSIZE -- size of the command history file
        	\item HOSTFILE -- file path of the hostname completion
        \end{itemize}
      \subsection{o}
        \begin{itemize}
        	\item OLDPWD
        \end{itemize}
      \subsection{p}
        \begin{itemize}
        	\item PWD
        \end{itemize}
    
    \section{Mechanism}
      \subsection{configuration of locate database}
        \begin{itemize}
        	\item /etc/updatedb.conf
        \end{itemize}
      \subsection{Shell Startup Configuration}
        \begin{itemize}
        	\item Interactive Shells
        	  \begin{itemize}
        	  	\item Login Shells
        	  	  \begin{itemize}
        	  	  	\item You can tell if a shell is a login shell by running \emph{echo \$0 ;} if the first character is a - , the shell’s a login shell.
        	  	  	\item The order of the execution of startup files:
        	  	  	  \begin{itemize}
        	  	  	  	\item /etc/profile
        	  	  	  	\item Then it looks for a user’s .bash\_profile, .bash\_login, and .profile files, running only the first one that it sees.
        	  	  	  	\item Ubuntu hasn't \textasciitilde/.bash\_profile, but \textasciitilde/.profile instead. CentOS hasn't \textasciitilde/.profile, but \textasciitilde/.bash\_profile instead.
        	  	  	  \end{itemize}
        	  	  \end{itemize}
          	    \item Non-Login Shells
          	      \begin{itemize}
          	      	\item A non-login shell is an additional shell that you run after you log in. It’s
          	      	simply any interactive shell that’s not a login shell.
          	      	\item \emph{bash} runs \emph{/etc/bash.bashrc} or \emph{/etc/bashrc} and then runs the user’s \emph{.bashrc}
          	      	\item Ubuntu hasn't \emph{/etc/bashrc}, but \emph{/etc/bash.bashrc} instead. CentOS hasn't \emph{/etc/bash.bashrc}, but \emph{/etc/bashrc} instead.
          	      \end{itemize}
        	  \end{itemize}
        	\item Noninteractive Shells -- Don't read any startup files
        \end{itemize}
    
    \section{Command Examples}
      \subsection{brace expansion}
        \begin{itemize}
        	\item touch memo\{1,2,3,4,5\}
        	\item ls \{a..f\}\{1..5\}
        \end{itemize}
      \subsection{run unaliased commands}
        \begin{itemize}
        	\item \verb|\rm| dir --- \emph{run the unaliased rm command}
        	\item The backslash in front of the command causes any command to run unaliased.
        \end{itemize}
      \subsection{chmod}
        \begin{itemize}
        	\item chmod -R 755 \$HOME/myapps
        	\item chmod 000 file
        	\item chmod ug+rx files
        	\item chmod go-rwx file
        \end{itemize}
      \subsection{chown}
        \begin{itemize}
        	\item chown joe /home/joe/memo.txt
        	\item chown joe:joe /home/joe/memo.txt
        	\item chown -R joe:joe /media/myusb
        \end{itemize}
      \subsection{cp}
        \begin{itemize}
        	\item cp -a ...
        	\item cp -i ...
        \end{itemize}
      \subsection{locate}
        \begin{itemize}
        	\item locate -i ...
        	\item updatedb
        \end{itemize}
      \subsection{ls}
        \begin{itemize}
        	\item ls -aF
        	\item ls -t
        	\item ls -S
        	\item ls \verb|--|hide=g*
        	\item permission bits
        	  \begin{itemize}
        	  	\item -rwsr-xr-x
        	  	\item drwxrwxr-t
        	  	\item -rw-rw-r\verb|--|+
        	  \end{itemize}
        \end{itemize}
      \subsection{mv}
        \begin{itemize}
        	\item mv -i ...
        \end{itemize}
      \subsection{rm}
        \begin{itemize}
        	\item rm -i ...
        \end{itemize}
      \subsection{umask}
        \begin{itemize}
        	\item fully opened permissions for a file is 666 and for a directory is 777
        	\item the real initial permission is the fully opened permission substracts the argument of \emph{umask}
        	\item umask
        	\item umask 022
        \end{itemize}
      \subsection{vi}
        \begin{itemize}
        	\item Ctrl+d
        	\item Ctrl+g
        	\item Ctrl+u
        	\item /The.*foot
        	\item ?[pP]rint
        	\item y\verb|)|
        	\item y\}
        \end{itemize}
\end{document}